\documentclass{article}

\usepackage{listings}

\title{Python}
\author{Mohammadali Varfan}

\begin{document}
\maketitle
content list here \\*
python syntax \\*
numpy and scipy suntax \\*
ploting \\*
loading data \\*
testing\\*
machine learning \\*
ide tips

\part{Introduction}

\part{Python}
\section{Basics}

\subsection{Variables}
In python we have 5 data types: string, number, tuple, list, dictionary. String and number are simple data types and each can hold just one entity but the other three can hold many items inside themselves.

\subsubsection{String and Number}
\begin{lstlisting}

# Defining a string
stringvar = "Hello"

# Defining an integer
num = 5

# Defining a floating variable
fltnum = 5.0

\end{lstlisting}

\subsubsection{List}
Lists are the most popular python data structure. They can store a collection of items with different types. We can define a list in many ways and for accessing its items we can use square brackets [] and also use slice operator [:] for accessing many items as below:
\begin{lstlisting}[language = Python]
"""
Defining a list with different syntaxes
"""
varlst = [5, 9, "hello", 5.687, "MAS", 99]
varlst2 = list()  # creates an empty list
varlst3 = list((9, 5.68, 1, 66))


# accessing list elements
print varlst  		# prints all the list
print varlst[0]  	# prints the first element at zero position
print varlst[-1]  	# prints the last element
print varlst[1:5]  	# prints all elements from first position to fifth
					# but not including the fifth
print varlst[2:]  	# prints all elements from second position to the end
\end{lstlisting}

\subsubsection{Tuple}
Tuples are

\subsubsection{Dictionary}
\subsection{Comments}
Comments in python:
\begin{lstlisting}[language=python]
# Single line comment

"""
Multi
line
comment
"""
\end{lstlisting}

\subsection{Naming Conventions}
Python naming conventions:
\begin{itemize}
\item Class names start with an uppercase letter. All other identifiers start with a lowercase letter.
\item Starting an identifier with a single leading underscore indicates that the identifier is private.
\item Starting an identifier with two leading underscores indicates a strongly private identifier.
\item If the identifier also ends with two trailing underscores, the identifier is a language-defined special name.
\end{itemize}



\section{Control Flow}
\section{Loops}
\section{Data Structures}
\section{Function}
\section{OOP}

\part{numpy and scipy}
\part{Plotting and Loading Data}
\part{Testing}
\part{Machine Learning}
\part{IDE tips}


\end{document}
